\documentclass{article}\usepackage[]{graphicx}\usepackage[usenames,dvipsnames]{color}
%% maxwidth is the original width if it is less than linewidth
%% otherwise use linewidth (to make sure the graphics do not exceed the margin)
\makeatletter
\def\maxwidth{ %
  \ifdim\Gin@nat@width>\linewidth
    \linewidth
  \else
    \Gin@nat@width
  \fi
}
\makeatother

\definecolor{fgcolor}{rgb}{0.251, 0.251, 0.251}
\newcommand{\hlnum}[1]{\textcolor[rgb]{0.816,0.125,0.439}{#1}}%
\newcommand{\hlstr}[1]{\textcolor[rgb]{0.251,0.627,0.251}{#1}}%
\newcommand{\hlcom}[1]{\textcolor[rgb]{0.502,0.502,0.502}{\textit{#1}}}%
\newcommand{\hlopt}[1]{\textcolor[rgb]{0,0,0}{#1}}%
\newcommand{\hlstd}[1]{\textcolor[rgb]{0.251,0.251,0.251}{#1}}%
\newcommand{\hlkwa}[1]{\textcolor[rgb]{0.125,0.125,0.941}{#1}}%
\newcommand{\hlkwb}[1]{\textcolor[rgb]{0,0,0}{#1}}%
\newcommand{\hlkwc}[1]{\textcolor[rgb]{0.251,0.251,0.251}{#1}}%
\newcommand{\hlkwd}[1]{\textcolor[rgb]{0.878,0.439,0.125}{#1}}%
\let\hlipl\hlkwb

\newenvironment{knitrout}{}{} % an empty environment to be redefined in TeX
\usepackage{alltt}

\RequirePackage[]{/home/arvind/.R/x86_64-pc-linux-gnu-library/3.4/BiocStyle/resources/tex/Bioconductor2}
\AtBeginDocument{\bibliographystyle{/home/arvind/.R/x86_64-pc-linux-gnu-library/3.4/BiocStyle/resources/tex/unsrturl}}


\title{The Xeva user's guide}
\author[1,2]{Arvind Mer}
\author[1,2,3]{Benjamin Haibe-Kains}
\affil[1]{Princess Margaret Cancer Centre, University Health Network, Toronto Canada}
\affil[2]{Department of Medical Biophysics, University of Toronto, Toronto Canada}
\affil[3]{Department of Computer Science, University of Toronto, Toronto Canada}

\date{\today}
\IfFileExists{upquote.sty}{\usepackage{upquote}}{}
\begin{document}
\maketitle
\tableofcontents
\newpage



\section{Introduction}

The Xeva package provides efficient and powerfull functions for patient drived xenograft (PDX) based pharmacogenomic data analysis.

\section{Installation and Settings}

Xeva requires that several packages are installed. However, all dependencies are available from CRAN or Bioconductor.

\begin{knitrout}
\definecolor{shadecolor}{rgb}{0.941, 0.941, 0.941}\color{fgcolor}\begin{kframe}
\begin{alltt}
\hlkwd{source}\hlstd{(}\hlstr{'http://bioconductor.org/biocLite.R'}\hlstd{)}
\hlkwd{biocLite}\hlstd{(}\hlstr{'Xeva'}\hlstd{)}
\end{alltt}
\end{kframe}
\end{knitrout}

Load Xeva into your current workspace:
\begin{knitrout}
\definecolor{shadecolor}{rgb}{0.941, 0.941, 0.941}\color{fgcolor}\begin{kframe}
\begin{alltt}
\hlkwd{library}\hlstd{(Xeva)}
\end{alltt}
\end{kframe}
\end{knitrout}

Load PDXE breast cancer dataset:
\begin{knitrout}
\definecolor{shadecolor}{rgb}{0.941, 0.941, 0.941}\color{fgcolor}\begin{kframe}
\begin{alltt}
\hlkwd{data}\hlstd{(brca)}
\hlkwd{print}\hlstd{(brca)}
\end{alltt}
\begin{verbatim}
## Xeva-set name: PDXE.BRCA
## Creation date: Mon Aug 27 11:16:00 2018
## Number of models: 849
## Number of drugs: 22
## Moleculer dataset: RNASeq, mutation, microArray, cnv
\end{verbatim}
\end{kframe}
\end{knitrout}

\section{Definations}
Before we further dive into the analysis and visualization, it is important to underastand terms used in the \Rpackage{Xeva} package.
In a \textbf{Xeva} object, the \textbf{experiment} slot stores each individual PDX/mouse data.
Other then the tumore growth data (time vs. tumor volume), for each individual PDX/mouse we can have meta data such as patient's age, sex, tissue histology, passage infromation etc.
All this data is stored using the class \textbf{pdxModel} and a unique id called \texttt{model.id} is given to each PDX/mouse model. We will see later how to get data for an individual \textit{model.id}.




\end{document}
